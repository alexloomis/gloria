\documentclass[12pt,letterpaper]{article}
\usepackage[utf8]{inputenc}
\usepackage[english]{babel}
\usepackage{amsmath,amsfonts,amssymb,amsthm}
\usepackage{mathtools}
\usepackage{mathpazo}
\usepackage[margin=2cm]{geometry}

\author{Alex Loomis}
\title{Notes}

\begin{document}

\maketitle

\paragraph{On contradictions:}

One present can have two pasts.
Ea created the world to fill it with her children.
When Zeus killed Chronos, his spilt blood created the world.
Osiris will end the world, jealous of the attention Isis bestows upon it.
The world will end when Coyote loses it in a game of cards.
The world always was, and always will be.
These are all true.
As reality diverges into hypotheticals and dreams,
so to does it converge, melding disparate pasts.

\tableofcontents

Note that pretty much everything (code, other notes, etc) supercedes this
in case of conflict.

Square grid, so armies can move straight towards each other with a straight front.

Game name ideas: eternal revolution, divine revolution, omniscience,
battle of the gods, ascension

Is it possible to make the game with no economy?
Like if instead of buying units,
they are recruited based on in-game achievements.
This would make it easier to avoid doom-stacks,
and units/nations could be balanced
based on how quickly the achievements can be completed.
This could also guide to relevant unlocks,
and lean into faction design.
For example, a faction with good cavalry
could get a leader with an effect beneficial to cavalry
after getting a certain number of cavalry kills.

Event based game: units are recruited by doing specific things,
pretenders are awakened by doing specific things,
maybe even each pretender has a specific ritual they must complete to ascend,
i.e.\ every pretender has a different win condition.
Fertility godess would win when global pop passes a certain number,
a death god when a certain number of units have died, etc.

Try to have more different lore per faction
by assigning a ``genre'' to each unit backstory?

Possible late-game acceleration:
as the pretenders absorb
the late omnipotence's essence,
still discharging into the universe,
they all grow closer to ascending.
As well, any unit which kills a pretender
could become significantly more powerful.

The game is non-elemental.
Though there may be creatures that live in water
or breathe fire, they are not water or fire units.
There can be casters with fire-based abilities,
but there cannot be pyromancers.

\section{Skills}

Skills are rolled against each other at 2d6 + attribute.

Idea: remove attributes that don't have `knowability'.
How many HP does Odin have? Who knows.
Use a wound system instead, or tie into immutability, or the like.

Why split physical/spell attacks?
Because \emph{spells aren't attacks}.
Attack is your ability to inflict damage,
spellcasting is your ability to inflict status effects.

Note that spell def defends against friendly spells too,
so highly defensive units are harder to buff.
Should spells increase spell def,
so the more you want to change a unit the harder it is to do?

\subsection{Attribute Levels}

Attribute levels typically range from one to six.
These represent

\begin{enumerate}
	\item someone unskilled,
	\item an ordinary person who relies on the skill,
	\item a highly capable person, an expert in the skill,
	\item legendary humans, weak pretenders,
	\item someone with beyond-mortal capabilities, capable but not an expert,
	\item someone with beyond-mortal capabilities working in their domain of expertise.
\end{enumerate}

\subsubsection{Spell Casting}

\begin{enumerate}
	\item Someone sensitive to magic, able to tap into it at sites of great power
	      (Andrew Ketterly, Rinsewind).
	\item An ordinary sorceror or witch (Nanny Ogg).
	\item An archmage, the head of a coven (Ged).
	\item Legendary casters (Morgan le Fay).
	\item Many pretenders (Odin).
	\item Pretenders whose domain is magic (Mystra).
\end{enumerate}

\subsubsection{Combat}

\begin{enumerate}
	\item Most humans.
	\item Soldiers and mercenaries.
	\item Champion duelists (Cyrano).
	\item Legendary heros (Gilgamesh).
	\item Many pretenders (Demeter).
	\item Pretenders whose focus is war (Mars).
\end{enumerate}

\subsection{List of Attributes}

\begin{itemize}
	\item Hit points
	\item Move
	\item Morale
	\item Attack
	\item Defense
	\item Valor (basically spell casting for fighters)
	\item Spell casting
	\item Magic resistance
\end{itemize}

\subsubsection{Morale}
Morale is resistance to fear effects, including routs due to low Hp.

\subsubsection{Valor}

Units gain enemy valor squared experience per kill, divided evenly,
and require (own valor + 1) squared experience to gain one valor.
Each point of valor provides an additional point of morale, attack, defense, and magic resistance.

Having enough valor allows, e.g.\ bersearkers to turn into bears.
As well, only a number of units equal to the max valor on the team can move each turn.
Should it be the square rood of the summed squares instead?

\section{Spell Casting}

Spell levels are roughly

\begin{enumerate}
	\item trivial effects,
	\item personal effects,
	\item local effects,
	\item battlefield-wide effects,
	\item province-wide effects,
	\item global or permanent effects,
	\item appocalyptic effects,
\end{enumerate}

though of course a strong enough personal effect
might be a higher level than a battlefield-wide effect, etc.

The highest level spell a caster can cast is
their level, plus the strength of magic in the province
(between -3 and +3).
Sacred casters and pretenders are capped as well by the devotion in the province.
Some mages have the ability to temporarily boost their level,
for example blood mages may sacrifice their life force to boost their casting.

High enough level spells can only ever be cast once.
There's no Ragnorack 2: the Electric Boogaloo.

Possible types of casters include
\begin{itemize}
	\item Summoners - summon units in battle, and ritual summons
	\item Druids - environmental manipulation
	\item Augers - rituals to scout and reveal information
	\item Sorcerers - blow stuff up. boom.
	\item Enchanters - enthrall people
\end{itemize}
In general, most spells do not do direct dammage.

\subsubsection{Magic Resistance}
In order to effect someone (friend or foe) with a spell,
the caster must win a spell casting vs spell resistance roll.

\section{Items}

Most units can cary up to a single magic item.
It is far easier to find items than it is to craft them.
Each magic item should be very impactful.

\section{Troops/Combat}

Each commander has a preset retinue that cannot generally be changed.
Losses replenish with time, faster in friendly territory.

The lower the current hitpoints of a unit,
the higher the chance of developing a permanent injury when damaged.

If meelee and ranged need balancing, try making meelee hit all enemies in range.

Idea for AI: unit and ability tags.
Each ability has tags indicating e.g.\ that it deals damage.
Particularly nuanced abilities can have a query function to ask it if/how it wants to be used.
Example: A mage unit has a fireball spell.
It has AoE and damage tags, so the AI will try to move to where it can hit multiple enemies.
A support unit has an ability that keeps nearby allies from dying.
It has the AoE, buff, and special tags, so it will be asked where to move.
If it has no preference, like if all units are at full health,
then it will move to be near the largest group of allies,
because of the buff and AoE tags.
The ability which is asking the hardest (has the best utility score) will be used.

Unit tags give global (to the unit) modifiers.
If a unit has the fearless tag, it would prioritize being near enemies,
regardless of the range of its abilities.

When one unit attacks another,
every unit with enough movement to reach the tile has the option to join.
If they do so, it depletes their movement.

\section{Provinces}

Province details include

\begin{itemize}
	\item Terain
	\item Modifiers
	\item Buildings
	\item Neighbors
	\item Population
	\item Fealty (who controlls)
	\item Dominion (who they folllow)
	\item Devotion (how strongly they follow)
\end{itemize}

\section{Tribes}

Some questions for inspiration:
\begin{itemize}
	\item How active is the pretender?
	      Does he manifest to his followers,
	      or talk to them only through priests or a prophet?
	\item How is the religion organized?
	      Is there an organized hierarchy,
	      or are religious leaders anyone people will listen to?
	\item What religious organizations exist,
	      like missions or monastaries?
	      If there is a religious hierarchy,
	      which of these organizations are affiliated with the official structure?
\end{itemize}

Remark that if a Pretender grants his powers to a priesthood,
their own, direct power is diminished.
So particularly paranoid or narcissistic pretenders
should have generally mundane clergy.
Or possibly only particularly trusting and generous pretenders
share their power.

Design idea: for each tribe, choose a mental illness their pretender embodies?
Should create dramatic, conflict-prone designs.

\subsection{The Ak'taar}

\paragraph{Blurb}

The members of this cult believe there are terrible ancient beings
sleeping beneath the ocean whose breathing causes the tides.
They make human sacrifices to appease them and to keep them asleep.

\paragraph{Pretender}

The pretender gains an ability that allows it to sacrifice victims to gain power,
and sacrifice its power to keep ancient evils imprisoned.
Each turn the ancient evils progress towards breaking free.
Once they break free, they wander around destroying provinces,
biased in favor of the pretender's land and dominion.

\paragraph{Characters}

(Spy) XXX is a heretic, secretly working to liberate the evil.
He may be used to accelerate its release.

\subsection{The Oshu}
\paragraph{Blurb}

The Oshu bond for life with a spirit when they turn 13.
Twins are sacred, and are bonded with each other.

\paragraph{Characters}

(Warrior) XXX and YYY are brother and sister, famed for their hunting prowess.
XXX hunts with a bow, while YYY hunts in the guise of a fox.

(Mage) XXX and YYY are childhood friends.
XXX is a druid, whereas YYY is an augur.
YYY is skilled at detecting spirit bonds,
and may help units that lack one to find theirs.

(Warrior) XXX and YYY served in an army together.
They are known for their their ability to magically enhance their
speed, strength, to appear from nowhere, etc.

\paragraph{Pretender Abilities}

The Oshu pick two pretender avatars,
which awaken at the same time.

\subsection{The Kairoth}
\paragraph{Blurb}

The Kairoth are expert sea men.
They revere knowledge and are frequently capable of some level of augery.

They believe the world was created of water swallowed by the Unnamed Fish God,
and will end when the water has left the fish's stomach.

\paragraph{Characters}

\subsection{The Duru}
\paragraph{Blurb}

The Duru worship trees.
The strongest and wisest trees come to life to protect the villagers.
The very wisest among the trees have an ability to grant wishes.

\paragraph{Pretender}

Pretender gains the wish ability?
Wishes of the form ``TAG [GETS LOSES] STATUS.''

\paragraph{Characters}

\subsection{The Ilun}
\paragraph{Blurb}

The people of Ilune began worship of their God-King
after he returned from the land of death.
Fanatics sacrifice themselves in his name
so that he may harness their power to hold his death at bay.
They live in the desert, so their hydromancers are particularly skilled.

\paragraph{Characters}

\paragraph{Units}

The Ilune have priests with the blood link ability.
As long as they are on the field,
all damage to the pretender is redirected to them.

\subsection{Not Satanists}
\paragraph{Blurb}

Believes the world was made by a selfish creator in order to have subjects to rule.
A host of angles banded together to overthrow him.
The war rages on, the angels now locked in an eternal struggle to protect humanity.

\paragraph{Characters}

(Mage?) Jayamati Simha (change) was sent
by the Rebellious Host to advise an ancient king.
Upon the king's death it was not summoned back to heaven,
so it now wanders the planet.

(Warrior) XXX is an ancient king whose reign saw
the tribe's kingdom expand to half a continent.
He was legendary for his strength and combat prowess.
He died when he was betrayed by his wife, stealing his strength.
Even with his strength gone,
he was able to kill 400 men before his enemies could land a killing blow.

(Mage) XXX is a prophet who lives deep in the jungle
and speaks directly with angels.

\paragraph{Pretender Abilities}

The XXX's pretender gains divine purity,
which prevents the addition of further status effects.

\subsection{Not Amazons}
\paragraph{Blurb}

This cult consists entirely of women.
They revere a goddess of change (the moon/seasons/similar)
who gives them shape-shifting abilities.
She leads a large number of non-human followers in her cult,
including amaroks, tariaksuq, and bears.

\paragraph{Characters}

(Leader) Big Bear, a massive kodiak bear, is the high priestess of the cult.
She does not talk, even when in human form, because bears don't talk.
Nonetheless, her guidance has lead her people to unprecedented strength.

(Mage) If Big Bear is the brains behind the cult, then XXX is the face.
no one knows her original form, she has never shifted back out of
the radiantly beautiful human form she took upon after joining.
She serves as ambassador to other groups, leading negotioations,
and also leads any public rituals the cult performs.

(Spy) XXX the Tariaksuq leads the cult's recruiting efforts,
She sneaks into cities, looking for women who are "unhappy with their lot",
and offers them a chance at a better life.

\paragraph{Pretender}

Choose a second avatar while making the pretender.
The pretender can switch to this body and back at will.

\subsection{Solipsia}
\paragraph{Blurb}

\emph{Much like the future, the past is unknowable, and inconsequential.}
The Solipsians believe the world to be illusory.
As such, they are very capable illusionists.
Their soldiers are fearless,
as they believe they already walk the land of the dead (i.e.\ our land).
% as death in this world is but waking in the other.
They are skeletons.

\paragraph{Characters}

\paragraph{Pretender Abilities}

The XXX's pretender gains scattered reality.
At the start of combat create four illusory copies of the pretender.
They may freely exchange places with any of their illusions,
including from other abilities, during their turn.
If an illusion is attacked, it is immediately destroyed.

\paragraph{Units}

This unit creates an illusion of himself.
If he wishes to dissimilate the true him,
he can exchange places with the illusion.

\subsection{Animal Peeps}
\paragraph{Blurb}

This tribe is cursed.
In order to break the curse,
they treat every living creature with reverence.

\paragraph{Characters}

\subsection{Clay Guys}
\paragraph{Blurb}

The residents of this village are protected by
giant clay bodies burried around the village.

\paragraph{Characters}

\section{Pretenders}

Pretenders are built from three parts,
their tribe, avatar, and aspect.
Each tribe has a limited number of avatars to choose from,
and each avatar has a limited number of aspects.
Names are avatar, aspect, and then tribe,
as in Persephone, Lady of the Underwold, Godess of the Greeks.

Beneficial abilities should only be able to target believers,
the nation's native troops.

Avoid superset abilities, e.g.\ lesser/greater thing.
Builds should go all in, or not in at all.

Abilities should never be able to be toned down.
If a god-like being shows up to destroy shit,
you're going to get a god-like level of destruction.

It should be impossible (or very very difficult) to kill a pretender
without specially preparing for it.

\begin{enumerate}
	\item As One -- so long as any of your units have hitpoints,
	      none of them can die
	\item Awe -- this unit cannot be directly targeted
	\item Beguile -- convert a unit and turn it into a believer
	\item Beautiful -- any eyes which see this pretender
	      refuse to see anything else ever again
	\item Dyad -- this pretender has a second avatar (with its own aspect)
	      they may switch to at will
	\item Gather Pain -- gain stats per dead ally (or per nearby dead unit?).
	\item Lucky -- always get max rolls
	\item Perfection -- no one can measure up to this pretender,
	      they always have perfect stats (all are +6es)
	\item Protector -- this pretender intercepts any attack on a believer
	      which would have been fatal
	\item Pure -- this pretender cannot gain status effects % too similar to awe?
	\item Reaper -- gain ability that instantly kills target
	\item Scattered Reality --
	      At the start of combat create four illusory copies of the pretender.
	      They may freely exchange places with any of their illusions,
	      including from other abilities, during their turn.
	      If an illusion is attacked, it is immediately destroyed.
	\item Share Pain -- transfer damage to nearby allies
	\item Teleportation -- allows instant movement to any area % too similar to scattered reality?
	\item Twins -- this pretender has a twin, an identical avatar.
	      Aspects may differ. % it is fitting that there be two twin abilities
\end{enumerate}

The pretenders begins in exile,
imprisoned by the old omnipotence.

This pretender was put in an enchanted slumber,
rather than imprisoned.
Their faction may sacrifice units, theirs and others', to awaken him.
Each month, he wakes up if, e.g., 2d6 + total devotion sacrificed > 2d6 + 50.

The Ugly God  was once beautiful,
but repeatedly betrayed the gods to bring gifts to humanity,
being disfigured in a new way each time as punishment.

\begin{enumerate}
	\item Holy tapestry that's carried around,
	      like the ark of the covenant.
	      Buffs nearby believers?
	\item Flying heavenly city.
	      Believers that die in battle \emph{to an enemy}
	      ascend to the city and give it strength.
	\item Shapeshifter, can transform into any creature.
	      Include creatures not found elsewhere in the game
	      to show off how many forms he can take, and add flavor to the world?
	      Portrait is randomly chosen each time it's shown.
\end{enumerate}

This pretender is a dead god, a former omniscience (i.e.\ a previous winner).
He reaches into the mortal realm to effect his resurrection,
and to ensure his return to his rightful place.

A skeletal patron of growth and fertility,
who sacrifices his essence to bring life to the world.
Maybe minus max health for plus economy?
Portrait should go from woman to skeleton as ability is used.

\section{Items}

\begin{enumerate}
	\item Runegold
\end{enumerate}

\clearpage

\section{Scenarios}

\subsection{First scenario/tutorial}
You are the old omniscience.
Kill as many usurpers as you can before being overwhelmed.

\paragraph{Prompt.}
Your servants have rebelled,
taken your power as their own,
declared themselves gods.
Their numbers overwhelm you,
but the scantest fraction of your power blows them as dust into eternity.
Yet, as millennia pass, even your strength may falter,
succumb to their unending tide.
Do not yield, but unmake as many usurpers and their allies
as your remaining divinity allows.
Their treachery shall not be rewarded.

% \paragraph{Teaches.}
% How to move a unit,
% how to use abilities.
%
% \paragraph{Description.}
% The omnicience is trapped in his now-ruined palace.
% There are many units on the map,
% including many pretenders.
% Each turn more spawn.
% The fewer remain, the more spawn
% and the more powerful they are.
% The player controls the omnicience.
% His abilities include unravel reality
% (instantly kills anything within a radius around him)

\paragraph{Win condition.}
The omnicience falls below a health threshold.

% \paragraph{Score tracked.}
% Number of kills.
% Legendary heros count for one point,
% userpers for ten.

\subsection{Second scenario/tutorial}
You and your allies have gravely wounded the ancient omnicience.
He must die.

\paragraph{Prompt.}
You have weakened the omniscience.
Yet even weak, even flailing blindly in pain,
the omniscience is more powerful than ten of the so-called new gods.
Wave after wave of gods and wave after wave of their champions
crash against his might.
Now, five hundred years after his first sign of weakness,
his pace has slowed.
Reality does not unravel in his presence.
Mortals may gaze upon him without going blind.
He is close to death.
Finish him.

% \paragraph{Teaches.}
% How to move leaders with cohorts.
% How to manage multiple units.
% How to use abilities.
%
% \paragraph{Description.}
% The omnicience is trapped in his now-ruined palace.
% There are many units on the map,
% including many pretenders.
% Each turn more spawn.
% The fewer remain, the more spawn
% and the more powerful they are.
% The player controls the usurpers.

\paragraph{Win condition.}
The omnicience dies.

% \paragraph{Score tracked.}
% Turns taken.

\subsection{Third scenario/tutorial}
You have slain the omnipotence,
and absorbed much of his power.
Fend off your rivals.

\paragraph{Prompt.}
You are [NAME1], Slayer of the omnipotence.
The world [rejoices at your might/trembles before your power].
As the omnipotence's energies scatter to the world, it rejuvenates,
[comes to life once more with followers to worship you/%
		teems with life ready to pass to your realm].
Yet, all is not certain.
Your allies of a moment ago are now rivals for this newly won power.
Chief among them is [NAME2].
Defeat them, and secure your place in the coming age.

% \paragraph{Teaches.}
% The immediate precursor era to the game's setting.
% Tactical combat with more than one enemy force.
% The use of terrain features.
% Potentially that terrain features are destructible.
%
% \paragraph{Description.}
% Player is given control of the unit which landed the killing blow.
% It becomes a pretender if it is not already.
% The player controls the pretender and their cohort on a tactical map.
% There are many forces,
% but other than the player and their target,
% there are no pretenders.
% Or, all other pretenders are severely weakened.
% Standing near the omnipitence's body
% greatly boosts a unit's power.

\paragraph{Win condition.}
Kill [NAME2].

% \paragraph{Score tracked.}
% Number of units surviving at scenario end.

\subsection{Fourth scenario/tutorial}

You have gravely wounded your enemies.
They hide nearby licking their wounds.
Kill them all.

\paragraph{Prompt.}
Not every rival has the decency to die.
Some scatter to the winds,
lick their wounds,
and build their forces against you.
They would wait until you are weak,
and strike out to replace you.
This cannot be tolerated.
Root them out, wherever they may be,
and show them the price of defiance.

% \paragraph{Teaches.}
% Army movement on strategic map.
% Interaction between tactical and strategic maps.
%
% \paragraph{Description.}
% Player controls a single army.
% Moves it around the map to attack other pretenders' armies.

\paragraph{Win condition.}
Kill three pretenders located in different provinces.

% \paragraph{Score tracked.}
% Number of units surviving at scenario end?
% Turns taken?

\subsection{Fifth scenario/tutorial}
Expand your reach across the continent.

\paragraph{Prompt.}
You have killed all those who dared oppose you,
and rewarded your loyal believers.
You are the undisputed god of your lands.
Yet in distant realms,
other pretenders have done the same.
Build your strength,
and invade their lands.
Prove that you alone deserve godhood.

% \paragraph{Description.}
% Starts a new game on a tiny map,
% starting with the pretender and units from the previous scenario.

\paragraph{Win condition.}
Same as normal to win a game.
Kill all other pretenders,
or control sufficiently many holy sites,
or force all other pretenders to declare fealty to you.

\end{document}
